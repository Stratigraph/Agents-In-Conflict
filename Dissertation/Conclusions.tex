\chapter{Conclusions}

Agent-based models have not been in the mainstream of international relations scholarship. Despite this, I believe that ABMs are uniquely suited to capture and model the complex interactions characterizing the international system, which I argue are often driven as much by uncertainty and satisficing heuristics as by formal rationality. In this dissertation, I set out to accomplish two overall goals: to adapt prior, well-established formal models of international political conflict into agent-based models, and to demonstrate how such models can be used comparatively to study the underlying theories they are based on. In doing so, I have incorporated elements from across the methodologies used in the study of international conflicts, including game theory, psychological and organizational decisionmaking theories, econometric models and machine-coded event data.

I did not initially set out to address the work of Bruce Bueno de Mesquita in particular; however, he is the key contributor to the two models this dissertation has concentrated on. The first is the International Interaction Game, presented in \citet{bdm_1992}: a formal, extensive-form model of how pairs of states choose whether to escalate their use of force against one another in a potential crisis situation. I chose this model since it is one of the few game-theoretic models that has been operationalized to make specific predictions about observed interactions, and as such is frequently cited as the best-tested application of the purely rational decisionmaking model of state decisionmaking \citep[e.g.][]{allison_1999,green_1996}. The second model is Bueno de Mesquita's Expected Utility model, a model of multilateral negotiations and conflict described across numerous papers \citep{bdm_1984,bdm_1994,bdm_1997,bdm_2002,bdm_2011} and popular presentations \citep{bdm_2009,bdm_2010}. This model is of interest not just because of the impressive accuracy attributed to it in the prediction of real-world cases \citep{bdm_2009,feder_1992}, but also because it is another relatively rare example of a formal, strategic model applied to forecasting and historic explanation.

Both of these models were, in practice, heavily computational already. The IIG model was tested most thoroughly via the EUGene software \citep{bennett_2000,bennett_2000b}, which not only performed the data manipulation required to combine several disparate datasets into a single million-record dataset, but applied a genetic algorithm to estimate each state's risk tolerance in each year by comparing its observed alliance portfolio to the most- and least-secure sets of alliances it could possibly hold. The Expected Utility model (EUM) was also implemented computationally -- in fact, the challenges of addressing it was that many of its elements are not well-articulated in the published descriptions.

What drew me to these models, however, was that both consist of unitary actors directly interacting with one another, making discrete decisions in sequence until a conclusion is reached. In this way, they both already shared a core structure with agent-based models\footnote{In fact, as I discuss in more detail in Chapter 3, the Expected Utility model may be best considered as an ABM already, despite being generally described as a game-theoretic model.}. Furthermore, the actors are not abstract `players' but represent specific real-world entities, with their choices, and the overall model outcomes, corresponding to events which may be observed in historic data. This offers the opportunity to not only reimplement these models as ABMs, but to directly compare their outputs against both the outputs of the original models and the empirical outcomes of the cases and interactions being modeled.

Both models can be separated into several sub-models, governing when and how the agents will interact with one another, what choices they face when they do, and how they decide between those choices. It is the latter component I was particularly interested in. Both models originally implement particular theories of state decisionmaking: perfect rationality in the case of the IIG, and a heuristic-based rationality in the case of the EUM. Implementing different decisionmaking models allows me to directly compare the explanatory power of their underlying theories, holding the rest of the model structure fixed.

This comparative approach draws on a broad trend of comparative modeling in the study of international relations. Direct comparison of statistical model results is, of course, exceedingly common. To name a few other examples I have drawn on directly, \citet{allison_1999} apply several qualitative theories of state decisionmaking to the case of the Cuban Missile Crisis, while \citet{kaufmann_1994} compares the predictions of the rational model to psychological theories of individual decisionmaking in an analysis of German policy in the 1905-06 Morocco Crisis. The \citet{axelrod_1980} Prisoner's Dilemma tournament involved the comparison of multiple decisionmaking models controlling agents' play in repeated prisoner's dilemma games, and was motivated at least in part by the study of international conflict and cooperation. Most relevant, \citet{stokman_1994b} present two models, an early version of the EUM, and the \citet{stokman_1994} cooperation-focused Logrolling Model (which is described explicitly as a computational, object-oriented model), both of which implement different theories of collective decisionmaking. They instantiate these models with the same inputs, and compare their explanatory and predictive power with regard to European Community collective decisionmaking.

Chapter 2 described the International Interaction Game in detail: both the bilateral, extensive-form game itself and the methodology that \citet{bdm_1992} lay out, and \citet{bennett_2000b} apply, using the Correlates of War alliance network data to estimate the payoffs associated with each terminal node of the tree for each dyad of countries. It also described a simplified variant of the game, drawn from \citet{signorino_1999}, which was used for testing and experimentation purposes. It presented an agent-based architecture built around these games, where the games themselves are a sub-model governing the interaction of pairs of agents, from a set of agents that persist across an entire model instantiation; the model instantiation object itself sequences these interactions -- at random in the simplified model, and following the EUGene-generated historic dyads for the IIG variant.

Most importantly, I describe three models of agent decisionmaking that control how agents choose an action at each node of the tree. The first is simply full, formal rationality: agents have full access to the payoffs of each of the game tree's terminal nodes, and use backwards induction to find the best move at each node. When two such agents interact with each other, they will arrive at the equilibrium outcome. Next, I describe two learning models where the agents store and update weights on each possible move, which they use to make decisions stochastically. Both are built on a stochastic decision model, experience-weighted attraction, which has been shown to be both an effective machine learning technique and a model of how individuals and groups actually make decisions. The first learning model is the Reinforcement Learning (RL) model, where agents maintain a single set of move weights, which are updated based on the payoff each move leads to. This offers a stylized model of the Organizational Behavior view of decisionmaking \citep{allison_1999}, with the agents developing standard operating procedures, which they apply without considering whether they are well-suited to the specific interaction. The next learning model is a Case-Based Learning (CB) model, where the agents maintain a library of past interactions, and associated weights learned from each. This model captures the way \citet{march_1993} describe organizations as making decisions by retrieving rules associated with relevant experiences, and specifically the tendency of decisionmakers, and states more broadly, to address new crises by analogy to previous ones \citep{khong_1992,schuman_1992}.

I first apply all three decisionmaking models to the simplified interaction model, populating it with sets of heterogeneous agents, whose properties were set stochastically. Despite starting with no initial weights, the agents are capable of learning to play in ways that result in equilibrium outcomes more often than we would expect strictly by chance. Over multiple instantiations, neither the RL nor the CB model show a robust advantage over the other in collective learning of the equilibrium outcomes. When the agent properties are static, the CB model robustly (though not universally) outperforms the RL model, since it allows agents to learn the best-response strategies for each other agent. However, when the agent properties (and hence the equilibria) vary from step to step, the correct moves vary as well, making case-based learning less effective, and the RL model becomes more likely to lead to equilibrium interactions.

Next, I apply these models to the International Interaction Game itself with the historic dyads generated by the EUGene software, with one agent per state in the international system. Again, despite starting with no information, and despite the added complexity of the game tree, the agents collectively and robustly learn to play such that the equilibrium outcome is reached more often than randomly. Unlike with the simplified model, however, the frequency of equilibrium outcomes between instantiations is not distributed evenly: there are several apparent attractors, equilibrium frequencies which are converged to more often than by chance; furthermore, changes to the international system itself lead the equilibrium frequency to change sharply at similar points across the majority of instantiations. These change-points correspond to major shocks to the international system which are exogenous to this model, particularly the ends of the world wars. Finally, the outcomes of the RL and CB models have comparable power to the equilibrium outcomes in predicting the actual observed events between each country pair in each year. Some runs of both, in fact, converge to substantially better fits than the equilibria. 

Case studies such as \citet{allison_1999} and \citet{kaufmann_1994} show that non-rational accounts of decisionmaking can explain the results of particular historic interactions as well or better than a purely rational account of decisionmaking; however, such individual case studies are difficult to conduct at scale, and \citet{achen_1989} and others have argued that they are insufficiently standardized and constrained to use for testing general theories. These results suggest that agent-based models can serve as a method of testing non-rational decisionmaking theories across a large number of cases, and anchoring such tests to well-established formal and statistical methodology. In fact, these results appear to confirm the conclusions of \citet{allison_1999}, \citet{kaufmann_1994} and others, and indicate that non-rational models can offer similar explanatory power to the rational model, when applied to the same data. The learning models presented here produce the satisficing decisions that are a well-understood feature of organizational behavior \citep{march_1993} but are not present in the purely rational model. Furthermore, these models offer an endogenous explanation for apparent errors that is richer than simply adding an exogenous noise factor to agents' utilities \citep{signorino_1999}. Finally, the results I present here may help offer one reason for the disagreements between advocates of rational and non-rational views of decisionmaking. The models suggest that lessons and procedures learned from experience will, over time, tend to drive to decisions and behaviors that lead to better outcomes, which will with some frequency be the equilibrium outcome. This means that the equilibrium can in fact be a useful predictor of the outcome we will observe, even if the actors are not reaching it in a procedurally-rational way. Furthermore, when the actors deviate from the equilibrium, these deviations are not solely random mistakes, and may be driven by the application of lessons learned from prior interactions.
% Cite some more case studies?


The Expected Utility Model (EUM), which I explore in Chapter 3 and apply in Chapter 4, is intended to capture a different level of detail than the International Interaction Game. Where the form and payoffs of the IIG are specifically designed (as the name implies) with country-to-country interactions in mind, the EUM is intended to serve as a general-purpose model of formal or implicit negotiation. While the model is often presented as an direct application of game theory, its agents are not formally rational in the traditional sense, and the model itself is not studied analytically in parameterized form, but applied and solved computationally on a case-by-case basis. In this way, it already appears to be an agent-based model. Furthermore, for the most part, the EUM's sub-models are not direct operationalizations of specific qualitative theories. The model's overall assumptions are drawn from the Realist paradigm, but the model is itself a particular theory of state interactions. The main argument put forward for its correctness is its accuracy -- its ability to make correct predictions as to the outcomes of particular negotiations or political contests. One of the main pieces of evidence for its accuracy is the \citet{feder_1992} evaluation of the model's application by the CIA, claiming up to 90\% accuracy across multiple cases, though providing minimal details on the methodology used to initialize the model runs or determine what constitutes a correct prediction. Several other papers \citep[e.g.][]{bdm_1984,bdm_1997} and general-audience accounts \citep{bdm_2010} provide single-issue forecasts or predictions of past events using data further in the past. More recently, \citet{bdm_2011} conducted a multi-case evaluation of a modified variant of the model, using the \citet{thomson_2006} dataset of European Union issues and decisions, showing that it has better predictive power than both a simple non-strategic median voter model and the original EUM.

For my purposes, the most significant of the model's reported successes is its apparent prediction of the course and conclusion of the Cold War between the United States and the Soviet Union \citep{bdm_1998}, which I discuss in detail in Chapter 4. The importance of this claim is twofold: most obviously, it would show that the EUM can capture not just discrete issues but states' overall alignment in the international system, and how that alignment changes over a timeframe of multiple decades. It would also, in addition, offer evidence that the unexpected end of the Cold War was an emergent phenomenon arising from the complex interplay of interactions which are themselves well-understood by existing theories, and not an argument against those theories as \citet{gaddis_1992} and others have claimed. If that is the case, it would strengthen the case that agent-based models are an effective tool for the study of international relations.

In Chapter 3, I present my reimplementation of the EUM, breaking it down into its different sub-models and highlighting its agent-based character using the \citet{grimm_2006} Overview, Design concepts, and Details (ODD) structure. I attempt to identify and discuss the underlying assumptions and hypotheses embedded in these sub-models, and propose several alternative sub-model implementations. In particular, I highlight one core assumption of the original EUM: that when two actors engage in a direct conflict, the loser will adopt the winner's position. This assumption represents a substantially different view of coercion than described by \citet{schelling_1966} and others.

In both Chapters 3 and 4, I attempt to reproduce results of the EUM for three cases where both the inputs and at least representative outputs are provided: in Chapter 3 the automobile standards from \citet{bdm_1994} and the Chinese democratization model from \citet{bdm_2002}, and in Chapter 4 the Cold War model \citep{bdm_1998}. In none of these cases do I perfectly replicate the originally-published results. This is cause for some concern, as it suggests that I have failed to properly implement the described description, or that the original model was not implemented as described (with this `or' not being exclusive). To paraphrase \citet{washington_1796}, in reviewing my code I am unconscious of intentional error, I am nevertheless too sensible of my defects not to think it probable that I may have included many errors. The difficulty of reproducing agent-based models from their descriptions is well known \citep{axtell_1996,rand_2007,edmonds_2003}, and all the more so when the original papers do not themselves describe the model in computational terms. Furthermore, I verify that the \citet{scholz_2011} reimplementation reproduces the results of the automobile standards model output, but not the Chinese democratization one -- and that the reproduction of the former relies on a probability calculation which appears to contradict not only the form of the equation provided in multiple papers, but its basic underlying interpretation. This, along with the evidence that the Cold War model utilized a proprietary implementation that includes features not described in the open literature \citep{dii_2011}, means that we cannot necessarily take the original results as ground truth either. However, if the theory of interaction and decisionmaking articulated by the EUM provides an accurate model of reality then it should not be oversensitive to specific elements of its implementation, including those that differ between my implementation and the original. Indeed, it appears to be the case that there is no single `original' model, and that the EUM has undergone multiple changes over time, yet is still described as the same basic model. In two out of the three cases I attempt to replicate, my implementation's results are similar but not identical to those presented in the original papers.

The main argument for the validity (and utility) of the original model is its predictive ability, and this must be the main test of my reimplementation as well. In cases such as the timeline for automobile standards, there is a single, quantifiable outcome which can be compared to the model's median position. In the case of broader international competition, however, there is no such outcome; instead, we must utilize the model's other outputs, and compare them to observed data. I apply several model variants to two test cases, at different time periods and scales. One is the Cold War case discussed above; I extend the analysis in \citet{bdm_1998} to quantitatively test the models' ability to predict both militarized conflicts and mutual-defense alliances. The other is a new attempt to predict the volume of conflict events collected in the International Crisis Early Warning System (ICEWS) dataset. To the best of my knowledge, this is the first application of an EUM-type model to such micro-level event data. In both cases, I instantiate the models using data from the Correlates of War project, providing input data which can be independently recreated without relying on expert elicitation.

I apply several model variants to each case, then instantiate and run each multiple times in order to produce ranges of outputs. I hypothesize that if a model is capturing some of the underlying decisions and interactions, the more frequently an output is produced, the more likely a corresponding event or relationship is to be observed in the data.  In both cases, the model outputs have only weak predictive power, though they are better than random. This relationship indicates that the models are in fact capturing at least some of the drivers of state behavior in the international system, particularly since the model is projecting the numerous separate issues driving the system onto a single dimension.

Since the historic data is fixed, I can directly compare the goodness-of-fit measures of the outputs of the different model variants to one another. Across both test cases, the baseline model variant, that attempts to be the most direct reproduction of the original model, has substantially less predictive power than the variant which implements several alternative sub-models. This variant showed the most predictive power on the ICEWS data, on the strength of which I applied it to the Cold War case, where it again generated a better fit than the baseline variant. Importantly, in the Cold War case, though the baseline variant produces a distribution of median-position outcomes which bears resemblance to the distribution shown in \citet{bdm_1998}, the actual distribution is driven by a model artifact, one-step shifts in the median position, rather than sustained transitions. The updated model, in contrast, generates plausible dynamics as well, with punctuated, stable transitions by groups of agents in the same direction.

By breaking the EUM into its constituent sub-models, I was able to explicitly test the assumptions each sub-model is based on. In particular, I was able to test the hypothesis that conflicts will lead the losing agent to change its position and adopt that of the winner. This stood out as questionable while I was developing my model reimplementation, not only because it represents a different view of coercion than is generally found in the literature \citep[e.g.][]{schelling_1966,bratton_2005}, but because it violated my conceptualization of the model architecture, and how it relates to reality: specifically, the strict separation between agents' external interactions and their internal decisionmaking, with agents only able to observe the latter. To test this hypothesis, I implemented a model variant where conflict does not lead the losing agent to change position. Across both the ICEWS and Cold War cases, model variants incorporating that new sub-model showed increased predictive power; this is particularly visible with the ICEWS experiments in Chapter 3, wherein this sub-model increased the power of the overall model across several variants with different combinations of other sub-models. 

Taken together, these results give reason for both caution and optimism with regard to applying the EUM to international relations. Despite being initially developed for negotiations over specific issues, the model is capable of generating outputs with a degree of non-random correspondence to observed events. This appears to be true across both the high-resolution ICEWS data and the lower-resolution MID data. Furthermore, the Cold War case provides information useful for predicting the presence of alliances as well as conflicts, evidence that the model is capturing multiple dimensions of the actual dynamics of the international system. However, the predictive power of these output data is nevertheless weak, as indeed we might expect from a relatively simple model that abstracts away most of the issues driving cooperation and conflict. 

%I have not yet explored my reimplementation models' predictive power with regard to single-issue negotiations, the realm in which the original reportedly achieved a 90\% accuracy rating. 
These  results strongly suggest that none of the model variants are likely to achieve the 90\% claimed issue-specific accuracy \citep{feder_1992,bdm_2002} in predicting the broad course of the international system. At present, these models are less accurate than the purely statistical models used to predict event data \citep[e.g.][]{brandt_2011,ward_2013}. Like those models, the EUM variants present an out-of-sample prediction, as they are instantiated with data strictly preceding the time period to which the output is compared. Unlike those models, however, the input data is taken from a strictly different dataset. Furthermore, while the statistical methodologies are generally applied to one specific conflict at a time, I have applied the EUM model variants to the entire core of the international system, leading to an inevitable loss of resolution, and hence of accuracy. A more significant difference is that the EUM models attempt to explicitly capture the decisionmaking and actions of individual actors, providing not just an outcome prediction but a comprehensible account of how the outcome was reached.

Indeed, this is true of the International Interaction Game models as well; no variant of either model presently is capable of fitting data better than the best statistical models. Yet, though prediction is an important test of the models and their theories, it is not their only objective in and of itself. Both classes of models I have presented here serve as tools for explicitly articulating, and testing, theories and hypotheses as to how states (and organizational actors more broadly) make decisions and interact with one another, either in crisis or merely competitive situations. Instantiating and running the models using notional, randomly-generated data can serve a similar purpose to analytical, closed-form examination of a simpler model, allows us to identify categories of model outputs, examine its general behaviors and properties across multiple specific instantiations, and estimate the uncertainty it captures or excludes. Applying empirical input data and running multiple instantiations of the models generates distributions of events and outcomes. If one model variant produces outputs which correspond better to the observed events, this would provide evidence that the assumptions underlying that variant are a better description of the real processes; in the case of the IIG models, where no variant produces obviously better predictions, this is evidence that the variants themselves remain approximately equally plausible.  However, \citet{bennett_2000} adds numerous additional control variables to the logistic regressions which are not present here, such as regime type and years of ongoing peace. Adding such controls to the regressions of model outputs against historic events may help separate their explanatory power and highlight whether one provides more useful information than the others. More broadly, incorporating the model outputs within more advanced statistical methodologies may allow us to identify which model is providing useful information not captured by other data sources and methods, as well as improving the forecasting power of the overall statistical models. Finally, it may be the case that different model variants are capturing different aspects of the underlying real-world decisionmaking and dynamics, and that the most predictive power is gained by an ensemble of multiple model variants applied to the same input data.

In this dissertation, I have implemented only a small number of the overall model variants one can imagine, and have by no means exhausted the universe of plausible models of decisionmaking. More variants provide a natural way of extending this research. One natural extension is a unification of the two model frameworks I have described here; \citet{bdm_2011} has already proposed the use of the IIG game tree in an EUM variant in order to better model scenarios (such as intra-European Union negotiations) where conflict is unlikely. More interesting, from my perspective, would be an integration of the types of decisionmaking models applied in both. The learning models used in the IIG, though well-grounded, do not lend themselves to straightforward interpretation, and it is not clear how to align the weighted stochastic choice to detailed qualitative and historical sources on the specific decisionmaking processes observed in particular conflicts. The type of heuristics used in the EUM variants would allow the testing of the rule-based decisionmaking that has been explored by expert systems \citep{taber_1992} and simpler \citep{cederman_1997} or narrower \citep{hudson_2004} agent-based models. The model architecture would allow different agents to utilize different sets of rules, and even for such rules to be developed via evolutionary computation \citep{axelrod_2006}. Meanwhile, the application of more rigorous and externally-grounded decisionmaking models to the EUM variants may help move it beyond a collection of effective heuristics. While the model's equilibrium may not be analytically tractable, learning agents may be able to approach it, or identify when an equilibrium does not exist \citep{galla_2013}. Recursive $n$-order rationality has proven to be an effective way of modeling strategic behavior, and is particularly applicable to modeling organizations \citep{latek_2009}; implementing it within the EUMs would provide a formal way of introducing bounded but non-heuristic rationality. There are additional artificial intelligence methods which can give the agents the ability to look multiple steps ahead and anticipate agents' responses to different courses of action. For example, Monte Carlo Tree Search \citep{chaslot_2008} and Markov decision process models more broadly, have proved to be powerful decisionmaking models in multi-agent simulations -- including in commercial computer games simulating international conflict and diplomacy \citep{champandard_2014}\footnote{In fact, the model architecture I utilize for the EUM allows agents to make internal copies of the model state, substituting their own decisionmaking model for those of the other agents, in order to facilitate look-ahead models such as Monte Carlo Tree Search or recursive rationality. However, implementing these models was outside the scope of this dissertation.}.
% RAX: Consider citing Luke et al. 2005 review of multiagent learning?

I chose to study the IIG and EUM in part because both are focused on modeling actors' strategic decisionmaking, to the exclusion of many other factors driving the international system. In particular, both models require that agents' capabilities be set exogenously: at instantiation in the EUM, and based on annual historic National Material Capabilities data in the IIG. In both models, the outcome of individual interactions and conflicts does not affect the agents' capabilities going forward. The plausibility of this assumption will vary from case to case, and conflict to conflict. While the use of non-military and `soft' power is unlikely to explicitly diminish it \citep{nye_1990}, the same is not necessarily true for military force, where factors as mundane as ammunition stocks can change the force an actor can plausibly bring to bear. While most of the EUM variants I describe limit agents to at most two conflicts per step (one which they initiate, one initiated against them), the IIG incorporates no such limit, and allows states (and implicitly their allies, who contribute capability as well) to apply their full power to any number of conflicts simultaneously. Yet when a state mobilizes its power (particularly military, though arguably other forms as well) to credibly threaten or deter another state, this is highly likely to reduce the power it can credibly bring to bear simultaneously elsewhere. Even the United States, at the height of its `unipolar' power in the post-Cold War world, required an explicit doctrine and careful planning in order to credibly claim to be capable of fighting two wars simultaneously \citep{qdr_2006}. Many other models of international systems \citep[e.g.][]{axelrod_1997,cederman_1997,min_2002,taylor_2008} endogenize agents' power, and in particular have it diminish when it is used in conflict. While the EUM endogenizes agents' position changes (indeed, this is the main purpose of the model), the IIG does not. While incorporating additional dynamics would add more moving parts to the models which would need to be justified, verified and validated, they would also allow the models to produce richer notional histories with each run, generating more outputs for validation or prediction. More importantly, they would add additional constraints and tradeoffs for the decisionmaking models to address, potentially leading to richer models and capturing the effects of long-term planning which may be occurring in reality but is not incorporated into the models at the moment.

This dissertation has also examined variants of only two models, out of many existing ones, and more which may be proposed and developed. The same comparative modeling methodology should be straightforward to apply to additional models as well. For example, \citet{tsebelis_1988} and \citet{metternich_2013} describe and analyzes formal models of multilateral coalition formation within single countries (France and Thailand, respectively) and show that the model equilibria have predictive power for real events. It should be possible to agentize these models as well, endow the agents with different decisionmaking models, and analyze the resulting outputs as I have done here. Furthermore, it is not necessary to begin with a formal model. Agent-based models such as \citet{cederman_1997} are natural subjects for this methodology as well, providing opportunities to implement both alternative heuristics and learning-driven decisionmaking models. 

While there are many opportunities for future work, I have accomplished the main goal I set out to achieve. In this thesis, I have restated two well-established models used in the study of international relations, and reimplemented them as agent-based models. In doing so, I highlighted the implicit model of state decisionmaking the models had previously incorporated, and proposed and implemented alternatives. By directly comparing the models to one another, to the originals, and to empirical data, I was able to test them, identify where they converge and differ, and test their underlying theories. This research demonstrates that ABMs can serve as a bridge between methodologies in international relations, building on formal models to incorporate behavioral insights derived from case studies and comparatively testing them using empirical data. While I have applied this methodology to international relations in particular, it can likely be applied across other areas of social science as well. ABMs may serve as a way of building upon existing methodological tools, and bridging between them: operationalizing qualitative theories and intuitions, articulating underlying assumptions and proposing alternatives, and then comparatively testing them against empirical data in order to assess their external validity. In fact, it may turn out that in many fields, as I argue is the case in international relations, models which look very much like ABMs are already widely accepted, but simply not identified as such.