

% ================================================================================
% ================================================================================
% VV OLD VV
% ================================================================================

In ``The Essence of Decision'' \citep{allison_1999} writes \emph{`` ''}. The book does just that: it describes three theories of political decisionmaking -- rational choice, organizational process, and and governmental politics -- and demonstrates their strenghts and weaknesses in analyzing one particular case study of political decisionmaking, the 1962 Cuban Missile Crisis. This example was not chosen at random. It involved the world's two preeminent superpowers at the time, the United States and the Soviet Union, deciding whether to initiate and escalate a military confrontation that had the real possibility of resulting in a nuclear war. Even without the nuclear stakes, decisions of whether to engage in a use of force against another state -- potentially, to go to war -- is among the most consequential external decisions that a state can make. Much of international relations theory, particularly in the Realist schools \citep{keohane_1986}, focuses on states' abilities to exert force on one another and deter the exertion of force against them. 

The view of states as rational actors is often a key part of such theories. While many (e.g. \citet{} and \citet{allison_1999}) discuss rationality in terms of qualitative reasoning, more formal methods have also been used to model, explain, or even plan state behavior. Much of modern game theory was developed to study international conflict \citep{gates_1997}, and it provides a powerful tool for formally studying the implications of rational, interactive behavior \citep{snidal_1985}. Much of game-theoretic modeling in international relations is relatively abstract, used to identify general principles rather than study particular empirical situations. One important exception is included in ``War and Reason'' \citep{bdm_1992}, which presents a game-theoretic, extensive-form model of how states decide whether or not to escalate their use of force when facing a potential conflict with another state. While most of the book is devoted to conventional formal analysis of the model in order to identify the equilibria under different parameters, the authors also attempt to test the theory empirically. They propose a methodology to estimate the game's utilities from empirical data for any historic country dyad. Using these utilities, they find the game equilibria for a subset of historic European country-dyads. They show that these equilibria are useful predictors of the correspinding events observed between those dyads, from the maintenance of the status quo to armed conflict.

While it is not presented as such, the \citet{bdm_1992} model resembles an agent-based model in many ways: it consists of distinct, unitary actors with well-specified decision rules, carrying out discrete interactions with one another in sequence. Unlike the majority of agent-based models in international relations, such as \citet{axelrod_1997,cederman_1997,min_2002}, it is driven by data and seeks to be explicitly explanatory and even predictive of specific events. Additionally, unlike many agent-based models in general, many of the model's dynamics are not determined by the agents themselves: long-term changes in the actors' capabilities and alliances are driven purely by the data, and are not affected by the results of particular interactions.

A key feature of the interaction between countries is the concept of `escalation'. When a crisis arises, either exogenously or initiated by one of the actors, both parties have iterated opportunities to increase the level of force or threat of force they are bringing to bear until the crisis is resolved. If both actors commit their entire military strength (or at least a substantial subset of it), the outcome is war. If neither actor makes a demand against the other, the crisis is resolved (or does not occur to begin with) and the status quo is preserved. There exists a range of outcomes between these two extremes, for example with one state threatening force against a weaker state, which in turn yields to the stronger state's demand rather than risk a war it knows it will lose.

As the last example implies, states' behaviors in such crises is generally strategic -- that is, each state anticipates the other party's response to any potential action, and chooses the action to take accordingly. As \citet{signorino_1999,smith_1999} demonstrate, this strategic behavior leads to interdependence between the actor properties and the outcomes of a given crisis which are not amenable to linear modeling and traditional econometric estimation. Indeed, escalation between states (or other comparable actors) is oftened modeled as an extensive-form game, with players choosing to escalate or not at each node until an outcome is reached based on how far each player was willing to escalate. 

A key assumption that frequently goes along with modeling inter-state crises as extensive-form games is rationality. States are assumed to attach a certain utility to each potential outcome of the game tree, and use their own utilities and those of the other state to find the subgame-perfect equilibrium, and play (that is, escalate or not) at each node of the tree accordingly. Deviations from this equilibrium are attributed to to an independent, normal error term either in states' perception of the relevant utilities \citep{signorino_1999}, or present in the overall model estimation \citep{smith_1999,bennett_2000}. Yet as \citet{allison_1999} and other have argued, rationality is only one model of state decisionmaking. Studies examining other models of decisionmaking have rarely, to the best of my knowledge, used the extensive-form game model. Rather, they have used detailed qualitative case studies \citep{achen_1989}, or (more rarely) traditional econometric techniques \citep{}. This makes it difficult to perform a direct comparison between the predictions of the different models when aggregated across many different cases.

Analysis such as \citet{allison_1999} demonstrates the power of directly applying different models of decisionmaking to the same historic event; however, the qualitative case-study methodology is time- and information-intensive, and may be prone to biases on the part of human coders \citep{}. It is worth noting that the Allison and Zelikow analysis relied extensively on primary sources which will not be available for the majority of cases.

Agent-based modeling offers a way to bridge this divide. Agents, representing states, can be paired up and play through an extensive-form game according to some internal behavior rules. These rules may implement perfectly rational behavior, purely random choice, or any other theory or model. While similar experiments have been conducted to evaluate the effectiveness of various strategies (e.g. the \citet{axelrod_1980} prisoners' dilemma tournament), the same procedure may be applied to real data. Much like \citet{bdm_1992} and \citet{bennett_2000} apply game-theoretic models to historic dyads, we may pair up agents with the desired decisionmaking rules based on the same dyads, once or many times, recording the results. Furthermore, if the agents' behavior changes over time (i.e. the agents are learning), this procedure allows us to capture the time-dependence between interactions which is not incorporated into the rational model. Finally, since the agent-based interactions produce the same output -- a terminal game tree node -- as the game-theoretic model, the agent-based model's output can be tested as a predictor of the historic observed outcome exactly the same as the game-theoretic model, allowing direct comparison between them, as well as between model variants utilizing different decision rules.

This paper is organized as follows. I first present the general architecture for an agent-based model of dyadic, extensive-form interactions with variable agent decisionmaking. I then present several behavior variants, grounding them both in political and multi-agent theory. I present a notional, simplified model international system, which I use to evaluate the limiting behaviors of the various decisionmaking rules. Finally, I apply these models to historic data, and evaluate their explanatory power.

